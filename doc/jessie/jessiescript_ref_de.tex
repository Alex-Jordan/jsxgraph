\documentclass[10pt]{article}
\usepackage{a4wide}
\usepackage[latin1]{inputenc}
\usepackage[ngerman]{babel}
\usepackage[ngerman]{varioref}
\usepackage[T1] {fontenc}
\usepackage[latin1] {inputenc}
\usepackage{amsmath}
\usepackage{amssymb}
\usepackage{verbatim}
\usepackage{pst-all}
\usepackage{float}
%\usepackage{pst-pdf}
\usepackage{multicol}
\usepackage{multirow}
\usepackage{tabularx}


\usepackage{geometry}
\geometry{a4paper,left=30mm,right=30mm, top=2cm, bottom=2cm}

\newcommand{\R}{\mathbb{R}}
\newcommand{\Q}{\mathbb{Q}}
\newcommand{\Z}{\mathbb{Z}}
\newcommand{\N}{\mathbb{N}}
\newcommand{\C}{\mathbb{C}}
\newcommand{\PC}{\mathcal{P}}
\newcommand{\ZC}{\mathcal{Z}}
\newcommand{\Sum}{\sum\limits}
\newcommand{\Prod}{\prod\limits}
\newcommand{\Lim}{\lim\limits}
\newcommand{\Int}{\int\limits}
\renewcommand{\Im}{\text{Im}}
\renewcommand{\Re}{\text{Re}}
\newcommand{\eps}{\varepsilon}
\parindent 0pt
\pagestyle{empty}


\newcommand{\luecke}[1]{\raisebox{-6pt}{\makebox[#1]{\dotfill}}}

\def\versionnumber{0.81}  % Version of this reference card
\def\year{2010}
\def\month{Juni}
\def\version{\month\ \year\ v\versionnumber}

\begin{document}
\title{JessieScript Referenz (Version \version)}
\begin{center} {\LARGE\textbf {JessieScript Referenz (Version
\version)}}
\end{center}
\section{Konstruieren}
Einfache mathematische JSXGraph-Konstruktionen k�nnen
mit dem Befehl
\begin{verbatim}
     board.construct(...);
\end{verbatim}
erzeugt werden. Dabei k�nnen verschiedene Elemente, durch
Semikolon getrennt als String �bergeben werden. Leerzeichen spielen keine Rolle. \\\\
M�gliche
Elemente sind: \\ \begin{tabular}{|l|l|} \hline \\[-0.75em] {\large
\textbf{Beispiel}} & {\large \textbf{Beschreibung}} \\
\hline\hline \verb+A(1,1)+ & Punkt an der Stelle (1,1) mit dem
Namen \verb'A'
\\ \hline \verb+BB(-2|0.5)+ & Punkt
an der Stelle (-2,0.5) mit dem Namen \verb'BB' \\
\hline\verb+]AB[+ & Gerade durch die Punkte \verb'A' und \verb'B' \\
\hline\verb+[AB[+ &
Halbgerade durch die Punkte \verb'A' und \verb'B', �ber \verb'B' hinaus \\
\hline\verb+]A BB]+ & Halbgerade durch die Punkte A
und BB, �ber A hinaus \\ \hline\verb+[AB]+ & Strecke zwischen \verb'A' und \verb'B' \\
\hline \verb+g=[AB]+ & Strecke zwischen \verb'A' und \verb'B' mit dem Namen \verb'g' \\
\hline\verb+k(A,4)+ & Kreis um \verb'A' mit Radius 4 \\
\hline\verb+k(A,[BC])+ & Kreis um \verb'A', dessen Radius durch
die L�nge der (nicht notwendigerweise
\\ & existierenden) Strecke \verb'[BC]' \\ \hline\verb+k(A,B)+ & Kreis um \verb'A',
dessen Kreislinie durch den Punkt \verb'B' geht \\
\hline\verb+k1=k(A,3)+ & Kreis um \verb'A' mit Radius 3 mit dem
Namen \verb'k1'
\\ \hline\verb+P(g)+ & Gleiter \verb'P' auf dem Objekt \verb'g' \\
\hline\verb+Q(k1,0,1)+ & Gleiter \verb'Q' auf dem Objekt \verb'k1'
mit den Koordinaten (0,1) \\ \hline \verb+g&k1+ & Schnittpunkt(e)
der Objekte \verb'g' und \verb'k1' \\ \hline \verb+S=g&k1+ &
Schnittpunkt(e) der Objekte \verb'g' und \verb'k1'.  \\ & Mehrere
Schnittpunkte
werden mit \verb'S'$_1$ und \verb'S'$_2$ bezeichnet, einzelne mit \verb'S'. \\
\hline\verb+||(A,g)+ & Parallele zur Geraden \verb'g' durch den
Punkt \verb'A'
\\  \hline\verb+|_(A,g)+ & Senkrechte zur Geraden \verb'g' durch den Punkt
A \\ \hline\verb+<(A,B,C)+ & Winkel, definiert durch die Punkte
\verb'A', \verb'B', \verb'C' \\ \hline\verb+alpha=<(A,B,C)+ &
Winkel, definiert durch die Punkte \verb'A', \verb'B', \verb'C',
mit dem Namen $\alpha$
\\  & M�gliche griechische Bezeichner sind \verb'alpha',
\verb'beta', \verb'gamma', \verb'delta', \verb'epsilon', \\ &
\verb'zeta', \verb'eta', \verb'theta', \verb'iota', \verb'kappa',
\verb'lambda', \verb'mu', \verb'nu', \verb'xi', \verb'omicron',
\verb'pi', \verb'rho', \verb'sigmaf', \\ & \verb'sigma',
\verb'tau', \verb'upsilon', \verb'phi', \verb'chi', \verb'psi' und
\verb'omega'. \\ \hline\verb+1/2(A,B)+ & Mittelpunkt von \verb'A'
und \verb'B' \\ \hline\verb+3/4(A,B)+ & Punkt, der die Strecke von
\verb'A' nach \verb'B' im Verh�ltnis 3:7 innen teilt, \\ & d.h.
$\frac{3}{4}$ der Strecke \verb'[AB]' liegen zwischen \verb'A' und
dem Teilpunkt \\ & Dabei ist jedes Verh�ltnis nat�rlicher Zahlen m�glich. \\
\hline\verb+P[A,B,C,D]+ & Polygon durch die Punkte \verb'A',
\verb'B', \verb'C', \verb'D' mit dem Namen 'P' \\
\hline\verb'f:x^2+2*x' & Funktionsgraph, $f:x\mapsto x^2+2\cdot x$ \\
\hline\verb'f:sin(x)' & Funktionsgraph, $g:x\mapsto \sin(x)$ \\
\hline\verb'#Hallo Welt(0,3)' & Text \verb'Hallo Welt' an den
Koordinaten (0,3) \\ \hline
\end{tabular} \vspace*{0.5cm} \\
Es ist f�r jedes der Elemente (au�er Punkte, Graphen und Polygone)
m�glich, mit
\begin{verbatim}
    objname = ...
\end{verbatim}
direkt einen Namen zu vergeben. \\ Die Funktion gibt ein Objekt
mit allen erzeugten Elementen zur�ck, sodass danach noch
Eigenschaften ver�ndert werden k�nnen.

\section{Schnelles Ver�ndern von Eigenschaften}
Zum Setzen der drei wichtigsten Eigenschaften gibt es eine
schnelle M�glichkeit, alle anderen m�ssen im Nachhinein durch
Zugriff auf die entsprechenden Objekte und Aufruf der
entsprechenden Methode gesetzt werden. \\ Diese sind \\
\begin{tabular}{|l|l|} \hline \\[-0.75em] {\large
\textbf{Eigenschaft}} & {\large \textbf{Beschreibung}} \\
\hline\hline\verb+invisible+ & das entsprechende Objekt ist unsichtbar \\
\hline\verb+draft+ & das entsprechende Objekt wird im
Entwurfsmodus
dargestellt \\
\hline\verb+nolabel+ & das entsprechende Objekt erh�lt kein Label
\\ \hline
\end{tabular} \vspace*{0.5cm} \\
Gesetzt werden diese Eigenschaften direkt beim Anlegen des
Objekts, indem das jeweilige Schl�sselwort (bzw. die jeweiligen
Schl�sselworte, auch eine Kombination davon ist m�glich), durch
Leerzeichen getrennt, hinter dem Konstruktionsbefehl noch vor dem
zugeh�rigen Semikolon, geschrieben wird, d.h.
\begin{verbatim}
     P(1,1) nolabel; Q(2,3) draft nolabel; [PQ] invisible;
\end{verbatim}

\section{Zugriff auf Elemente}
Der Zugriff auf die Elemente nach dem Konstruieren ist m�glich
�ber: \\
\begin{tabular}
{|l|l|} \hline \\[-0.75em] {\large
\textbf{Element}} & {\large \textbf{Beschreibung}} \\ \hline\hline
\verb+constr.points[i]+ & liefert den $i$-ten Punkt oder Gleiter
der Konstruktion \verb+constr+, \\ & dabei sind auch
Mittel- bzw. Teilpunkte unter den Punkten\\
\hline\verb+constr.lines[i]+ & liefert die $i$-te Gerade,
Halbgerade oder Strecke der Konstruktion \verb+constr+, \\ & dabei
sind auch Parallelen und Senkrechten unter den Geraden
\\
\hline\verb+constr.circles[i]+ & liefert den $i$-ten Kreis der
Konstruktion \verb+constr+
\\
\hline\verb+constr.intersections[i]+ & liefert den $i$-ten
Schnittpunkt der Konstruktion \verb+constr+
\\
\hline\verb+constr.angles[i]+ & liefert den $i$-ten Winkel der
Konstruktion \verb+constr+
\\
\hline\verb+constr.functions[i]+ & liefert die $i$-te Funktion der
Konstruktion \verb+constr+
\\
\hline\verb+constr.texts[i]+ & liefert das $i$-te Textelement der
Konstruktion \verb+constr+
\\
\hline\verb+constr.polygons[i]+ & liefert das $i$-te Polygon der
Konstruktion \verb+constr+
\\
\hline\verb+constr.A+ & liefert das Element mit dem Namen \verb'A'
der Konstruktion \verb+constr+ \\ \hline
\end{tabular}

\section{Makros}
Zus�tzlich k�nnen Makros definiert werden. Schl�sselwort ist
\verb+Macro+, die Parameter werden, durch Komma getrennt, in
runden Klammern �bergeben, der Inhalt innerhalb von geschweiften
Klammern. Links vom Zuweisungsoperator kann ein beliebiger Name
f�r das Makro �bergeben werden. \\ Die entsprechende Syntax ist
also
\begin{verbatim}
     macroName = Macro(param1, param2, param3, ...) { Befehl1; Befehl2; Befehl3; ... };
\end{verbatim}
Aufgerufen wird das Makro dann mit
\begin{verbatim}
     ergebnis = macroName(x1,x2,x3,...);
\end{verbatim}

\section{Beispiel}
Ein Beispiel soll zur Veranschaulichung der Anwendung dienen.
\begin{verbatim}
    board = JXG.JSXGraph.initBoard('box', {originX: 50, originY: 300, unitX: 50,
                                           unitY: 50, axis:true});

    cons1 = board.construct("A(1,1);BC(1,3);k(A,[BC]);X(2,4)");
    cons2 = board.construct("J(7,4);l_2=[BC A]");

    cons1.points[0].face('>'); // A
    cons1.BC.strokeColor('black');
    cons2.l_2.strokeWidth(4);
    cons1.X.size(8);

    cons3 = board.construct("test = Macro(D,E,F) { g=[DE] nolabel; k1=k(D,[EF]);};
                                 ttt=test(A,X,J);");
    cons3.ttt.g.strokeColor('red');
\end{verbatim}

\end{document}
