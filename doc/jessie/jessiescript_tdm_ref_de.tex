\documentclass[10pt]{article}
\usepackage{a4wide}
\usepackage[latin1]{inputenc}
\usepackage[ngerman]{babel}
\usepackage[ngerman]{varioref}
\usepackage[T1] {fontenc}
\usepackage[latin1] {inputenc}
\usepackage{amsmath}
\usepackage{amssymb}
\usepackage{verbatim}
\usepackage{pst-all}
\usepackage{float}
%\usepackage{pst-pdf}
\usepackage{multicol}
\usepackage{multirow}
\usepackage{tabularx}


\usepackage{geometry}
\geometry{a4paper,left=30mm,right=30mm, top=2cm, bottom=2cm}

\newcommand{\R}{\mathbb{R}}
\newcommand{\Q}{\mathbb{Q}}
\newcommand{\Z}{\mathbb{Z}}
\newcommand{\N}{\mathbb{N}}
\newcommand{\C}{\mathbb{C}}
\newcommand{\PC}{\mathcal{P}}
\newcommand{\ZC}{\mathcal{Z}}
\newcommand{\Sum}{\sum\limits}
\newcommand{\Prod}{\prod\limits}
\newcommand{\Lim}{\lim\limits}
\newcommand{\Int}{\int\limits}
\renewcommand{\Im}{\text{Im}}
\renewcommand{\Re}{\text{Re}}
\newcommand{\eps}{\varepsilon}
\parindent 0pt
\pagestyle{empty}


\newcommand{\luecke}[1]{\raisebox{-6pt}{\makebox[#1]{\dotfill}}}

\def\versionnumber{0.82}  % Version of this reference card
\def\year{2010}
\def\month{Juni}
\def\version{\month\ \year\ v\versionnumber}

\begin{document}
\title{JessieScript Referenz (Version \version)}
\begin{center} {\LARGE\textbf {JessieScript Referenz (Version
\version)}}
\end{center}
\section{Konstruieren}
Einfache mathematische JSXGraph-Konstruktionen k�nnen in der von
JSXGraph lesbaren Syntax JessieScript erzeugt werden. Dabei k�nnen
verschiedene Elemente, durch
Semikolon getrennt als String �bergeben werden. Leerzeichen spielen keine Rolle. \\\\
M�gliche
Elemente sind: \\ \begin{tabular}{|l|l|} \hline \\[-0.75em] {\large
\textbf{Beispiel}} & {\large \textbf{Beschreibung}} \\
\hline\hline \verb+A(1,1)+ & Punkt an der Stelle (1,1) mit dem
Namen \verb'A'
\\ \hline \verb+BB(-2|0.5)+ & Punkt
an der Stelle (-2,0.5) mit dem Namen \verb'BB' \\
\hline\verb+]AB[+ & Gerade durch die Punkte \verb'A' und \verb'B' \\
\hline\verb+[AB[+ &
Halbgerade durch die Punkte \verb'A' und \verb'B', �ber \verb'B' hinaus \\
\hline\verb+]A BB]+ & Halbgerade durch die Punkte A
und BB, �ber A hinaus \\ \hline\verb+[AB]+ & Strecke zwischen \verb'A' und \verb'B' \\
\hline \verb+g=[AB]+ & Strecke zwischen \verb'A' und \verb'B' mit dem Namen \verb'g' \\
\hline\verb+k(A,4)+ & Kreis um \verb'A' mit Radius 4 \\
\hline\verb+k(A,[BC])+ & Kreis um \verb'A', dessen Radius durch
die L�nge der (nicht notwendigerweise
\\ & existierenden) Strecke \verb'[BC]' \\ \hline\verb+k(A,B)+ & Kreis um \verb'A',
dessen Kreislinie durch den Punkt \verb'B' geht \\
\hline\verb+k1=k(A,3)+ & Kreis um \verb'A' mit Radius 3 mit dem
Namen \verb'k1'
\\ \hline\verb+P(g)+ & Gleiter \verb'P' auf dem Objekt \verb'g' \\
\hline\verb+Q(k1,0,1)+ & Gleiter \verb'Q' auf dem Objekt \verb'k1'
mit den Koordinaten (0,1) \\ \hline \verb+g&k1+ & Schnittpunkt(e)
der Objekte \verb'g' und \verb'k1' \\ \hline \verb+S=g&k1+ &
Schnittpunkt(e) der Objekte \verb'g' und \verb'k1'.  \\ & Mehrere
Schnittpunkte
werden mit \verb'S'$_1$ und \verb'S'$_2$ bezeichnet, einzelne mit \verb'S'. \\
\hline\verb+||(A,g)+ & Parallele zur Geraden \verb'g' durch den
Punkt \verb'A'
\\  \hline\verb+|_(A,g)+ & Senkrechte zur Geraden \verb'g' durch den Punkt
A \\ \hline\verb+<(A,B,C)+ & Winkel, definiert durch die Punkte
\verb'A', \verb'B', \verb'C' \\ \hline\verb+alpha=<(A,B,C)+ &
Winkel, definiert durch die Punkte \verb'A', \verb'B', \verb'C',
mit dem Namen $\alpha$
\\  & M�gliche griechische Bezeichner sind \verb'alpha',
\verb'beta', \verb'gamma', \verb'delta', \verb'epsilon', \\ &
\verb'zeta', \verb'eta', \verb'theta', \verb'iota', \verb'kappa',
\verb'lambda', \verb'mu', \verb'nu', \verb'xi', \verb'omicron',
\verb'pi', \verb'rho', \verb'sigmaf', \\ & \verb'sigma',
\verb'tau', \verb'upsilon', \verb'phi', \verb'chi', \verb'psi' und
\verb'omega'. \\ \hline\verb+1/2(A,B)+ & Mittelpunkt von \verb'A'
und \verb'B' \\ \hline\verb+3/4(A,B)+ & Punkt, der die Strecke von
\verb'A' nach \verb'B' im Verh�ltnis 3:7 innen teilt, \\ & d.h.
$\frac{3}{4}$ der Strecke \verb'[AB]' liegen zwischen \verb'A' und
dem Teilpunkt \\ & Dabei ist jedes Verh�ltnis nat�rlicher Zahlen m�glich. \\
\hline\verb+P[A,B,C,D]+ & Polygon durch die Punkte \verb'A',
\verb'B', \verb'C', \verb'D' mit dem Namen 'P' \\
\hline\verb'f:x^2+2*x' & Funktionsgraph, $f:x\mapsto x^2+2\cdot x$ \\
\hline\verb'f:sin(x)' & Funktionsgraph, $g:x\mapsto \sin(x)$ \\
\hline\verb'#Hallo Welt(0,3)' & Text \verb'Hallo Welt' an den
Koordinaten (0,3) \\ \hline
\end{tabular} \vspace*{0.5cm} \\
Es ist f�r jedes der Elemente (au�er Punkte, Graphen und Polygone)
m�glich, mit
\begin{verbatim}
    objname = ...
\end{verbatim}
direkt einen Namen zu vergeben.

\section{Schnelles Ver�ndern von Eigenschaften}
Zum Setzen der drei wichtigsten Eigenschaften gibt es eine
schnelle M�glichkeit, alle anderen m�ssen im Nachhinein durch
Zugriff auf die entsprechenden Objektnamen und Aufruf der
entsprechenden Methode gesetzt werden. \\ Diese sind \\
\begin{tabular}{|l|l|} \hline \\[-0.75em] {\large
\textbf{Eigenschaft}} & {\large \textbf{Beschreibung}} \\
\hline\hline\verb+invisible+ & das entsprechende Objekt ist unsichtbar \\
\hline\verb+draft+ & das entsprechende Objekt wird im
Entwurfsmodus
dargestellt \\
\hline\verb+nolabel+ & das entsprechende Objekt erh�lt kein Label
\\ \hline
\end{tabular} \vspace*{0.5cm} \\
Gesetzt werden diese Eigenschaften direkt beim Anlegen des
Objekts, indem das jeweilige Schl�sselwort (bzw. die jeweiligen
Schl�sselworte, auch eine Kombination davon ist m�glich), durch
Leerzeichen getrennt, hinter dem Konstruktionsbefehl noch vor dem
zugeh�rigen Semikolon, geschrieben wird, d.h.
\begin{verbatim}
     P(1,1) nolabel; Q(2,3) draft nolabel; [PQ] invisible;
\end{verbatim}

\section{Setzen von Eigenschaften}
M�chte man Eigenschaften der erzeugten Elemente im Nachhinein
ver�ndern, ist das auch per JessieScript m�glich. Die
entsprechende Syntax lautet
\begin{verbatim}
     objektname.eigenschaft = wert;
\end{verbatim}
Ein Beispiel w�re also
\begin{verbatim}
     A(1,2); A.size = 8;
\end{verbatim}
M�gliche Eigenschaften sind dabei \\
\begin{tabular}{|l|l|} \hline \\[-0.75em] {\large
\textbf{Eigenschaft}} & {\large \textbf{Beschreibung}} \\
\hline\hline\verb+strokecolor+ & Linienfarbe, entweder als englischer HTML-Farbname \\ & oder als Hex-Angabe \#rrggbb \\
\hline\verb+fillcolor+ & F�llfarbe, entweder als englischer HTML-Farbname \\ & oder als Hex-Angabe \#rrggbb \\
\hline\hline\verb+highlightstrokecolor+ & Linienfarbe w�hrend das Objekt hervorgehoben ist, \\ & entweder als englischer HTML-Farbname oder als Hex-Angabe \#rrggbb \\
\hline\verb+highlightfillcolor+ & F�llfarbe w�hrend das Objekt hervorgehoben ist, \\ & entweder als englischer HTML-Farbname oder als Hex-Angabe \#rrggbb \\
\hline\verb+labelcolor+ & Farbe des Labels, entweder als englischer HTML-Farbname \\ & oder als Hex-Angabe \#rrggbb \\
\hline\verb+strokewidth+ & Linienst�rke, in Pixel \\
\hline\verb+dash+ & Strichelung der Linie, m�gliche Werte sind dabei: \\
& 0: durchgezogene Linie \\ & 1: gepunktete Linie \\ & 2:
gestrichelte Linie mit kurzen Strichen \\ & 3: gestrichelte Linie
mit normalen Strichen \\ & 4: gestrichelte Linie mit langen
Strichen \\ & 5: gestrichelte Linie mit abwechselnd normalen und
langen Strichen \\ & \hspace{2.5mm} und gro�en L�cken \\ & 6:
gestrichelte Linie mit
abwechselnd normalen und langen Strichen \\ & \hspace{2.5mm} und kleinen L�cken \\
\hline\verb+visible+ & Objekt wird angezeigt (true) oder versteckt
(false) \\ \hline\verb+shadow+ & Objekt bekommt einen
Schatteneffekt (true) oder nicht (false)
\\ \hline\verb+size+ & (nur f�r Punkte) Gr��e des Punktes,
in Pixel
\\ \hline\verb+face+ & (nur f�r Punkte) Aussehen des
Punktes, m�gliche Werte sind dabei: \\ & Kreuz: \verb+cross+ oder
\verb+x+ \\ & Plus: \verb+plus+ oder \verb"+" \\ & Kreis:
\verb+circle+ oder \verb+o+ \\ & Quadrat: \verb+square+ oder
\verb+[]+ \\ & Diamant: \verb+diamond+ oder \verb+<>+ \\ & Dreieck
nach oben: \verb+triangleup+ oder \verb+A+ \\ & Dreieck nach
unten: \verb+triangledown+ oder \verb+v+ \\ & Dreieck nach rechts:
\verb+triangleright+ oder \verb+>+ \\ & Dreieck nach links:
\verb+triangleleft+ oder \verb+<+
\\ \hline
\end{tabular}

\section{Makros}
Zus�tzlich k�nnen Makros definiert werden. Schl�sselwort ist
\verb+Macro+, die Parameter werden, durch Komma getrennt, in
runden Klammern �bergeben, der Inhalt innerhalb von geschweiften
Klammern. Links vom Zuweisungsoperator kann ein beliebiger Name
f�r das Makro �bergeben werden. \\ Die entsprechende Syntax ist
also
\begin{verbatim}
     macroName = Macro(param1, param2, param3, ...) { Befehl1; Befehl2; Befehl3; ... };
\end{verbatim}
Aufgerufen wird das Makro dann mit
\begin{verbatim}
     ergebnis = macroName(x1,x2,x3,...);
\end{verbatim}
\end{document}
