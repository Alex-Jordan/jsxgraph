%%%%%%%%%%%%%%%%%%%%%%%%%%%%%%%%%%%%%%%%%%%%%%%%%%%%%%%%%%%
% Electronic Journal of Mathematics and Technology (eJMT) %
% style sheet for LaTeX.  Please do not modify sections   %
% or commands marked 'eJMT'.                              %
%                                                         %
%%%%%%%%%%%%%%%%%%%%%%%%%%%%%%%%%%%%%%%%%%%%%%%%%%%%%%%%%%%
%                                                         %
% eJMT commands                                           %
%                                                         %
\documentclass[12pt,a4paper]{article}%                    %
\usepackage{times}                                        %
\usepackage{amsfonts,amsmath,amssymb}                     %
\usepackage[a4paper]{geometry}                            %
\usepackage{fancyhdr}                                     %
\usepackage{color}                                        %
\usepackage[pdftex]{hyperref} % see note below            %
\usepackage{graphicx}%                                    %
\hypersetup{                                              %
    a4paper,                                              %
    breaklinks                                            %
}                                                         %
%                                                         %
\newtheorem{theorem}{Theorem}                             %
\newtheorem{acknowledgement}[theorem]{Acknowledgement}    %
\newtheorem{algorithm}[theorem]{Algorithm}                %
\newtheorem{axiom}[theorem]{Axiom}                        %
\newtheorem{case}[theorem]{Case}                          %
\newtheorem{claim}[theorem]{Claim}                        %
\newtheorem{conclusion}[theorem]{Conclusion}              %
\newtheorem{condition}[theorem]{Condition}                %
\newtheorem{conjecture}[theorem]{Conjecture}              %
\newtheorem{corollary}[theorem]{Corollary}                %
\newtheorem{criterion}[theorem]{Criterion}                %
\newtheorem{definition}[theorem]{Definition}              %
\newtheorem{example}[theorem]{Example}                    %
\newtheorem{exercise}[theorem]{Exercise}                  %
\newtheorem{lemma}[theorem]{Lemma}                        %
\newtheorem{notation}[theorem]{Notation}                  %
\newtheorem{problem}[theorem]{Problem}                    %
\newtheorem{proposition}[theorem]{Proposition}            %
\newtheorem{remark}[theorem]{Remark}                      %
\newtheorem{solution}[theorem]{Solution}                  %
\newtheorem{summary}[theorem]{Summary}                    %
\newenvironment{proof}[1][Proof]{\noindent\textbf{#1.} }  %
{\ \rule{0.5em}{0.5em}}                                   %
%                                                         %
% eJMT page dimensions                                    %
%                                                         %
\geometry{left=2cm,right=2cm,top=3.2cm,bottom=4cm}        %
%                                                         %
% eJMT header & footer                                    %
%                                                         %
\newcounter{ejmtFirstpage}                                %
\setcounter{ejmtFirstpage}{1}                             %
\pagestyle{empty}                                         %
\setlength{\headheight}{14pt}                             %
\geometry{left=2cm,right=2cm,top=3.2cm,bottom=4cm}        %
\pagestyle{fancyplain}                                    %
\fancyhf{}                                                %
\fancyhead[c]{\small The Electronic Journal of Mathematics%
\ and Technology, Volume 1, Number 1, ISSN 1933-2823}     %
\cfoot{%                                                  %
  \ifnum\value{ejmtFirstpage}=0%                          %
    {\vtop to\hsize{\hrule\vskip .2cm\thepage}}%          %
  \else\setcounter{ejmtFirstpage}{0}\fi%                  %
}                                                         %
%                                                         %
%%%%%%%%%%%%%%%%%%%%%%%%%%%%%%%%%%%%%%%%%%%%%%%%%%%%%%%%%%%
%
% Please place your own definitions here
%
\def\GEONExT{GEONE\kern-.06em \lower.5ex\hbox{x}\kern-.215em T}
%
%%%%%%%%%%%%%%%%%%%%%%%%%%%%%%%%%%%%%%%%%%%%%%%%%%%%%%%%%%%
%                                                         %
% How to use hyperref                                     %
% -------------------                                     %
%                                                         %
% Probably the only way you will need to use the hyperref %
% package is as follows.  To make some text, say          %
% "My Text Link", into a link to the URL                  %
% http://something.somewhere.com/mystuff, use             %
%                                                         %
% \href{http://something.somewhere.com/mystuff}{My Text Link}
%                                                         %
%%%%%%%%%%%%%%%%%%%%%%%%%%%%%%%%%%%%%%%%%%%%%%%%%%%%%%%%%%%
%
\begin{document}
%
% document title
%
\title{JSXGraph -- Dynamic Mathematics Running on (nearly) Every Device}%
%
% Single author.  Please supply at least your name,
% email address, and affiliation here.
%
\author{\begin{tabular}{c}
\textit{Michael Gerh\"auser, Bianca Valentin, Alfred Wassermann} \\
alfred.wassermann@uni-bayreuth.de\\
Department of Mathematics, 
University of Bayreuth\\
95440 Bayreuth, 
Germany\end{tabular}
}%
%
%%%%%%%%%%%%%%%%%%%%%%%%%%%%%%%%%%%%%%%%%%%%%%%%%%%%%%%%%%%
%                                                         %
% eJMT commands - do not change these                     %
%                                                         %
\date{}                                                   %
\maketitle                                                %
%                                                         %
%%%%%%%%%%%%%%%%%%%%%%%%%%%%%%%%%%%%%%%%%%%%%%%%%%%%%%%%%%%
%
% abstract
%
\begin{abstract}
%
Since Java applets seem to be on the retreat in web application, other 
approaches for displaying interactive mathematics in the web browser are needed. 
One such alternative could be our open-source project JSXGraph. It is a 
cross-browser library for displaying interactive geometry, function plotting, 
graphs, and data visualization in a web browser. It is implemented completely 
in JavaScript and uses the vector graphics formats SVG and VML. No further 
plug-ins are required.
%
\end{abstract}%
%
%%%%%%%%%%%%%%%%%%%%%%%%%%%%%%%%%%%%%%%%%%%%%%%%%%%%%%%%%%%
%                                                         %
% eJMT command                                            %
%                                                         %
\thispagestyle{fancy}                                     %
%                                                         %
%%%%%%%%%%%%%%%%%%%%%%%%%%%%%%%%%%%%%%%%%%%%%%%%%%%%%%%%%%%
%
% Please use the following to indicate sections, subsections,
% etc.  Please also use \subsubsection{...}, \paragraph{...}
% and \subparagraph{...} as necessary.
%
\section{Introduction}
JSXGraph is a free software library for displaying dynamic, graphical mathematics in a web browser.
Its feature set covers dynamic Geometry,  function graphs, curves, 
charts, and turtle graphics.

Usually, JSXGraph is embedded in web pages, for on- or off\/line viewing, the download size is a mere
80 kByte.
JSXGraph enhanced web pages can be viewed with all major web browsers on nearly every hardware platform and operating system.
The supported hardware ranges from smartphones and tablet computers running iOS or Android  to
Desktop PC running Windows, MacOS X or Linux.

At the time of writing, JSXGraph is the only dynamic geometry system that runs  on such a broad range of 
devices and web browsers---without installation of any plug-in or wathsoever additional software.
JSXGraph is usable even on devices with limited computing resources, like cheap tablet PCs or
outdated Desktop PCs running Microsoft Internet Explorer 6.0. 

Thus, this library may prove to be helpful for the
introduction of technology in mathematical education in developing countries.




JSXGraph is an open source project hosted by sourceforge, 
the library is released under the Lesser GNU General Public License (LGPL). 
In order to use JSXGraph the developer has to include only two files in the 
HTML file: the JSXGraph code and a CSS file. 

\bigskip
\hrule
\bigskip

The  size 
of the JSXGraph code is about 380 kByte. If the web server delivering the 
content has data compression enabled (which should be the default anyhow) the 
size of the transmitted code is about 80 kByte. To compare it with Java software, 
for example the size of the \GEONExT{} archive is about 1 Mbyte. JSXGraph does not 
rely on any other JavaScript library.

JSXGraph is not meant to be programmed directly by the teacher or student. Rather, 
it is a software library used transparently in a web page to display geometry 
resources, or to work internally as the mathematical visualization engine in a 
Web 2.0 application. For example the successor of \GEONExT{} will be based on 
JSXGraph. Nevertheless, for the dauntless teacher having some experience in 
JavaScript programming, it should be no problem to create  constructions with 
JSXGraph.


\section{Requirements}
JSXGraph runs on every hardware and operating system which has a graphical 
web browser. The range of supported hardware thus reaches from Desktop PCs 
down to tablet computers and smartphones.

All the mainstream web browser are supported, Firefox $3+$, Internet Explorer
6+ (including the upcoming version 9), Google Chrome (all versions). 
Also, the browsers Safari, Opera are supported since at least 2008. 

For smartphones the Opera mini is supported but without interactivity.
Also Android based devices are supported since the release of the JSXGraph v0.82.
The default browser on these devices (at least up to Android 2.2) does not provide
SVG or VML graphics. But in the latest version of JSXGraph 
the use of the HTML canvas element is enabled. Thus, a new range of devices is 
able to run JSXgraph.


\section{Features}
\subsection{Geometry}
Plane geometry with 
homogeneous or affine coordinates,



\subsection{Calculus and function plotting}
Function plotting, parametric curves, polar plots. 
Differential equation solver.

Interpolation: Lagrange interpolation, 
cubic splines, B-splines, Bezier curves.



\subsection{Other topics}
Projective transformations,
Turtle graphics, 
charting.
Initial attempts to display 3D points.



\subsection{Importing file formats}
    * \GEONExT{}
    * Intergeo file format i2geo
    * GeoGebra
    * Cinderella (alpha quality)
    * Arcview (server based) 
\subsection{Plug-ins}
    * moodle
    * wordpress
    * mediawiki
\begin{verbatim}
<jsxgraph width="500" height="500">
  var brd = JXG.JSXGraph.initBoard('jxgbox',{boundingbox:[-2,2,2,-2]});
  var p = brd.create('point',[1.5,1.5],{face:'o', size:8});
  brd.create('segment',[[0,0],p],{dash:3});
</jsxgraph>
\end{verbatim}                    
    * drupal 

\subsection{New features}
    * Bezier curves
    * Conic sections
    * \LaTeX{} syntax for labels and texts
          o ASCIIMathML (falls back to Google chart API)
          o MathJax (http://www.mathjax.org)
    * Animations
    * Flexible layer system 

\section{JessieScript}
    * Having to program everything with JavaScript to display math with JSXGraph is a hurdle for using it in classroom with students.
    * Alternative: JessieScript
    * JessieScript is a syntax similar to what is taught in schools and can be parsed by JSXGraph.
    * Examples: P(1,1); Q(-2,2); g=[PQ]; k(P,2); M=1/2(P,Q); ||(g,M);
    * Easy to learn and use
    * Fosters algorithmic thinking 




\section{Conclusion}

\begin{thebibliography}{99}
    \bibitem{geosvg} GeometryEditor (http://wme.cs.kent.edu/geosvg/), formerly known as GeoSVG.
    %\bibitem http://sourceforge.net 
    %\bibitem http://i2geo.net 
    \bibitem{sketchpad} http://www.dynamicgeometry.com
    \bibitem{cabri} http://cabri.com 
    \bibitem{cinderella} http://cinderella.de 
    \bibitem{geonext} http://geonext.de 
    \bibitem{geogebra} http://geogebra.org
    \bibitem{crockford} Crockford D. (2008) JavaScript: The good parts, Sebastopol, CA, O'Reilly.
\end{thebibliography}

\end{document}